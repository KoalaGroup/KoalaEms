\subsection{\path{camac/ral}}

Das Argument "`Slotnummer des Kontrollmodules"' in den folgenden Funktionen ist die
CAMAC-Slotadresse des RAL-Readout-Moduls. In einer sp"ateren Version wird
dies ein Index in die Adre"stabelle des Instrumentierungssystems sein.

Die Z"ahlung der "`RAL-Spalte"' beginnt mit Null.

\begin{procdesc}{RALconfig}{Ermittlung der Konfiguration des RAL-Auslesebaumes}
\args{Slotnummer des Kontrollmodules}
\descr{
	Die Routine ermittelt die Anzahl der RAL111-Chips in jeder Spalte des
	Auslesebaumes, indem feste Muster durch die aus den DAC- oder Test-Registern
	gebildeten FIFOs geschoben werden.
}
\data{Ein Datenwort pro Spalte, enth"alt die Anzahl der RAL111-Chips darin}
\rand{Die Hardware wird vollst"andig zur"uckgesetzt --- neue Initialisierung notwendig.}
\errors{Hardwarefehler}
\end{procdesc}

\begin{procdesc}{RALdatapathtest}{Test des 4-bit-Datenweges zum Frontend}
\args{Slotnummer des Kontrollmodules, RAL-Spalte, Anzahl der Testworte}
\descr{
	Ein aus einer aufsteigenden Rampe bestehendes Testmuster der gegebenen
	L"ange wird durch das aus den DAC- oder Test-Registern bestehende FIFO
	der gegebenen Spalte geschoben und das Resultat auf Konsistenz gepr"uft.
}
\rand{Die Hardware wird vollst"andig zur"uckgesetzt --- neue Initialisierung notwendig.}
\errors{Hardwarefehler}
\end{procdesc}

\begin{procdesc}{RALloadtestregs}{Laden der Test-Register in den RAL111-Chips}
\args{Slotnummer des Kontrollmodules, RAL-Spalte, Modus, [Daten]}
\descr{
	Die Test-Register (siehe "`RALtestreadout"') werden auf die gegebenen Werte
	gesetzt. Das "`Modus"'-Argument definiert die Art der Spezifikation:
	\begin{list}{}{}
		\item 0: Die durch folgende
			Datenargumente gegebenen Bitpositionen werden auf {\dq}1{\dq}
			gesetzt, alle anderen auf {\dq}0{\dq}.
		\item 1: Die durch folgende
			Datenargumente gegebenen Bitpositionen werden auf {\dq}0{\dq}
			gesetzt, alle anderen auf {\dq}1{\dq}.
		\item 2: Die folgenden Argumente werden als XDR-String interpretiert, der
			das komplette Bitmuster enth"alt. Die L"ange des Strings mu"s
			$N * 2$ sein, mit N als Anzahl der RAL111-Chips in der
			Spalte. (1 Chip enth"alt 16 bit)
	\end{list}
}
\rand{Das CAMAC-Readoutmodul wird zur"uckgesetzt.}
\errors{Argumentfehler, Hardwarefehler}
\end{procdesc}

\begin{procdesc}{RALloaddac}{Laden der DAC-Register in den RAL111-Chips}
\args{Slotnummer des Kontrollmodules, RAL-Spalte, Modus, [Daten]}
\descr{
	Die Threshold-DAC-Register werden auf die gegebenen Werte
	gesetzt. Das "`Modus"'-Argument definiert die Art der Spezifikation:
	\begin{list}{}{}
		\item 0: Es folgt 1 Datenwort; alle DAC-Register in der Spalte werden
			auf diesen Wert gesetzt.
		\item 1: Die folgenden Argumente werden als XDR-String interpretiert, der
			pro RAL111-Chip ein Byte enth"alt. Die L"ange des Strings mu"s
			der Anzahl der RAL111-Chips in der
			Spalte entsprechen. (Davon sind nur die niederwertigen 4 Bit
			signifikant.)
	\end{list}
}
\rand{Das CAMAC-Readoutmodul wird zur"uckgesetzt.}
\errors{Argumentfehler, Hardwarefehler}
\end{procdesc}

\begin{procdesc}{RALfilterramload}{Laden der Filter-Lookup-Table im CAMAC-Readoutmodul}
\args{Slotnummer des Kontrollmodules, Modus, [Daten]}
\descr{
	Der Filter-RAM wird mit den gegebenen Werten geladen.
	Das "`Modus"'-Argument definiert die Art der Spezifikation:
	\begin{list}{}{}
		\item 0: Die durch folgende
			Datenargumente gegebenen Bitpositionen werden auf {\dq}1{\dq}
			gesetzt, alle anderen auf {\dq}0{\dq}.
		\item 1: Die durch folgende
			Datenargumente gegebenen Bitpositionen werden auf {\dq}0{\dq}
			gesetzt, alle anderen auf {\dq}1{\dq}.
		\item 2: Die folgenden Argumente werden als XDR-String interpretiert, der
			das komplette Bitmuster enth"alt. Die L"ange des Strings mu"s
			16 kByte sein.
	\end{list}
}
\errors{Argumentfehler}
\end{procdesc}

\begin{procdesc}{RALfilterramread}{Auslesen der Filter-Lookup-Table im CAMAC-Readoutmodul}
\args{Slotnummer des Kontrollmodules}
\descr{
	Der Filter-RAM wird ausgelesen.
}
\data{Ein XDR-String der L"ange 16k wird erzeugt.}
\errors{Argumentfehler}
\end{procdesc}

\begin{procdesc}{RALfilterramtest}{Testen des Filter-RAMs im CAMAC-Readoutmodul}
\args{Slotnummer des Kontrollmodules}
\descr{
	Das Filter-RAM wird mit einem Testmuster beschrieben, zur"uckgelesen
	und verglichen.
}
\rand{Der Inhalt des Filter-RAMs wird "uberschrieben.}
\errors{Argumentfehler, Hardwarefehler}
\end{procdesc}

\begin{procdesc}{RALtestreadout}{Readout des RAL-Systems im Testmodus}
\args{Slotnummer des Kontrollmodules, Testmodus, STA1}
\descr{
	Das RAL-System wird in einem Testmodus ausgelesen. Das "`Modus"'-
	Argument kann folgende Werte annehmen:	
	\begin{list}{}{}
		\item 0: Die "`all-zero"'-Register in den RAL111-Chips werden
			ausgelesen. (Achtung --- die Chips k"onnen trotzdem Daten liefern,
			siehe Bemerkungen im Manual.)
		\item 1: Die "`all-one"'-Register in den RAL111-Chips werden
			ausgelesen.
		\item 2: Die Test-Register in den RAL111-Chips werden
			ausgelesen. (siehe "`RALloadtestregs"')
	\end{list}
	Das "`STA1"'-Argument bestimmt den Wert des STA1-Kontrollregisters
	in CAMAC-Readoutmodul (und damit "uber Zero-Suppression und Total-Readout ---
	siehe Manual).
}
\data{Der ausgelesenen Daten werden zur"uckgeliefert.}
\rand{Das CAMAC-Readoutmodul wird zur"uckgesetzt.}
\errors{Hardwarefehler}
\end{procdesc}

\begin{procdesc}{RALreadout}{Readout des RAL-Systems}
\args{Slotnummer des Kontrollmodules}
\descr{
	Das RAL-System wird ausgelesen und neu gestartet.
	Vorher m"ussen die Kontrollregister (durch einfache CAMAC-
	Kommandos) auf die gew"unschten Werte gesetzt worden sein
	und der Readout erstmalig gestartet werden.
}
\data{Der ausgelesenen Daten werden zur"uckgeliefert.}
\errors{Hardwarefehler (insbesondere, falls keine Daten ausgelesen
	werden konnten --- dann sollte die Zero-Suppression ausgeschaltet
	werden)}
\end{procdesc}

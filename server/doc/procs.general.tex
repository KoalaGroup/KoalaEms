\section{\path{general}}

\begin{procdesc}{Echo}{gibt alle Argumente aus}
\args{beliebig}
\descr{
	Alle Argumente werden unver�ndert in den Ausgabestrom geschrieben. Dabei wird keinerlei
	Hardware angesprochen, diese Funktion ist also z.B. zur Erzeugung von Testdaten brauchbar.
}
\data{alle Argumente}
\end{procdesc}

\begin{procdesc}{IdentIS}{gibt Informationen �ber das aktuelle Instrumentierungssystem aus}
\data{
	zuerst Anzahl der im Instrumentierungssystem definierten Module (siehe DownloadISModulList),
	dann paarweise jeweils Moduladresse und -typ
}
\rem{
	funktioniert auch, wenn keine Instrumentierungssysteme konfiguriert oder definiert sind (erzeugt
	dann die Ausgabe `0')
}
\end{procdesc}

\begin{procdesc}{Timestamp}{gibt Zeitmarke aus}
\data{3 Worte: Tage, Sekunden, Hundertstelsekunden}
\rem{
	Das Resultat ist im allgemeinen nicht absolut zu interpretieren, sondern nur zur Bildung von
	Differenzen brauchbar.
}
\end{procdesc}

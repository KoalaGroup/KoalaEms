\subsection{\path{camac/caennet}}

\def\caencontr{
	Die Prozedur arbeitet mit dem CAENnet-Controller C-139. (Das ist die
	"altere, langsamere Version.) Damit k"onnen die Hochspannungsger"ate
	SY 127 und SY 227 angesteuert werden.\\
}
\def\caenerr{
	Wenn ein CAENnet-Protokollfehler auftritt, wird \plerr{HW} zur"uckgeliefert
	und der vom CAENnet-Controller gelieferte Fehlercode im Datenblock abgelegt.
	Wenn der CAENnet-Controller nicht reagiert, wird nur \plerr{HW}  zur"uckgeliefert.
}

\begin{procdesc}{CAENnetRead}{
	liest einen Parameter oder eine Parameter-Gruppe aus einem
	CAENnet-Slave
}
\args{n (des Controllers), CAENnet-Crate-Nummer, Kanalnummer, Parameternummer}
\descr{
	\caencontr
	Die Kanalnummer kann sich auf einen tats"achlichen Kanal oder auch auf
	logische Gruppen (im Handbuch wird dies als "`Target"' bezeichnet) beziehen. Die
	Parameternummer kann einen einzelnen Wert oder auch einen
	Parameterblock bezeichnen.\\
	Diese Funktion liest alle f"ur die jeweilige Kanal-Parameter-Kombination
	gelieferten Daten aus. Das sind (f"ur ein SY127 $\rightarrow$ 40 Kan"ale mit
	je 14 Parametern) maximal $560 (+3 \hbox{Patternworte})$ Worte.
}
\data{Parameterwert(e)}
\errors{\caenerr}
\rem{
	siehe Beschreibung der Channel- und Parameternummern 
	in~\cite{hardware:caenc139}
}
\end{procdesc}

\begin{procdesc}{CAENnetWrite}
{schreibt einen Parameter in einen CAENnet-Slave}
\args{
	n (des Controllers), CAENnet-Crate-Nummer, Kanalnummer,
	Parameternummer, Parameterwert
}
\descr{
	\caencontr
	Die Kanalnummer kann sich auf einen tats"achlichen Kanal oder auch auf
	logische Gruppen (im Handbuch wird dies als "`Target"' bezeichnet) beziehen.\\
	Es wird ein einzelner Parameterwert f"ur den gew"ahlten Kanal bzw. die
	Gruppe geschrieben.
}
\errors{\caenerr}
\rem{
	siehe Beschreibung der Channel- und Parameternummern
	in~\cite{hardware:caenc139}
}
\end{procdesc}
